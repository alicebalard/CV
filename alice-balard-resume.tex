%%%%%%%%%%%%%%%%% PREAMBLE %%%%%%%%%%%%%%%%%%%%%%%%%%%%
%Change the font size of your document - 10pt, 12.1pt, etc.
\documentclass[letterpaper,11pt,oneside]{article}
\usepackage[utf8]{inputenc}
\usepackage{lmodern}
\usepackage[T1]{fontenc}
\usepackage{cmbright}
\usepackage{setspace}
\usepackage{hyperref}
\usepackage{fontawesome5} %for cool icons
\usepackage{multicol}
\usepackage[skins]{tcolorbox}

\hypersetup{
    colorlinks=true,
    linkcolor=blue,
    filecolor=magenta,      
    urlcolor=blue,
    }

\usepackage{graphicx}
\graphicspath{ {images/}} %upload your signature to this file
%Change the margins to fit your CV/resume content
\usepackage[left=.5in, right=.5in, bottom=.1in, top=.4in]{geometry}

%Changes the page numbers - {arabic}=arabic numerals, {gobble}=no page numbers, {roman}=Roman numerals
\pagenumbering{gobble}

%%%%%%%%%%%%%%%%% END OF PREAMBLE %%%%%%%%%%%%%%%%%%%%%

\begin{document}

%%%%%%%%%%%%%%%%% NAME OF APPLICANT %%%%%%%%%%%%%%%%%%%
\noindent  \Large{\textbf{Alice BALARD, PhD, DVM}} \\
\vspace{-3ex} 
\normalsize

%%%%%%%%%%%%%%%%% CONTACT INFORMATION %%%%%%%%%%%%%%%%%
\begin{multicols}{2}
  \begin{tcolorbox}[enhanced, width=\linewidth, left, size=fbox, 
                    fontupper=\large, drop shadow southwest, colback=teal!15!white]
    \textbf{Research fields and interests} \\
    Molecular ecology \& evolution \\
    Biostatistics - DNA methylation \\
    \faIcon{dna} multi-omics
  \end{tcolorbox}

  \begin{tcolorbox}[enhanced, width=\linewidth, left, size=fbox, 
                    fontupper=\large, drop shadow southwest, colback=teal!15!white]
    \faIcon{envelope} \href{mailto:alice.cam.balard@gmail.com}{alice.cam.balard@gmail.com} \\
    \faIcon{orcid} \href{https://orcid.org/0000-0002-0942-7479}{alice-balard-276b7974/}\\
    \faIcon{github} \href{https://www.github.com/alicebalard}
    {github.com/alicebalard}\\
    \includegraphics[height=1em]{butterfly.png}
 \href{https://bsky.app/profile/alicebalard.bsky.social}
    {alicebalard.bsky.social}
  \end{tcolorbox}
\end{multicols}

  \begin{tcolorbox}[enhanced, width=\linewidth, left, size=fbox, 
                    fontupper=\large, drop shadow southwest, colback=teal!15!white]
    \textbf{\faIcon{laptop}} R: data management, statistical analysis, -omics; Shell (bash): -omics; \LaTeX \\
    \faIcon{language} French (fluent), English (fluent), German (B2)
  \end{tcolorbox}
  
\vspace{.1em}

%%%%%%%%%%%%%%%%% MAIN BODY %%%%%%%%%%%%%%%%%%%%%%%%%%%
% The main body is contained in a tabular environment. To move sections onto the next page, simply end the tabular environment and begin a new tabular environment.

\noindent \begin{tabular}{@{} l p{0.95\linewidth}}

\Large{\faIcon{briefcase}}    & \textcolor{teal}{\textbf{PROFESSIONAL EXPERIENCE}} \\\vspace{-6pt}
     & \\
     & \textbf{Research Fellow (2024 - actual)} University College London (London, UK) \\
     & \textit{Interindividual DNA methylation variation} \\\vspace{-6pt}
     & \\
     & \textbf{Teaching Fellow (2023 - 2024)} Queen Mary University of London (London, UK) \\
     & \textit{Biology, bioinformatics, statistics} \\\vspace{-6pt}
     & \\
     & \textbf{Data Scientist (2023)} Biomodal (London, UK) \\
     & \textit{Customer collaboration for a new methylation sequencing technology} \\\vspace{-6pt}
     & \\
     & \textbf{Research fellow - Biostatistician (2023)} London School of Hygiene \& Tropical Medicine (London, UK) \\
     & \textit{Analyses of longitudinal data on Myalgic Encephalomyelitis / Chronic Fatigue Syndrome} \\\vspace{-6pt}
     & \\
     & \textbf{MSCA Individual Research Fellow (2021 - 2023)} Queen Mary University of London (London, UK) \\
     & \textit{DNA methylation inheritance in response to parasite infection in the three-spined stickleback} \\\vspace{-6pt}
     & \\
     & \textbf{PhD Student (2016 - 2021)} Humboldt University \& IZW (Berlin, Germany) \\
     & \textit{Resistance and tolerance to \textit{Eimeria} in the European house mouse hybrid zone} \\\vspace{-6pt}
     & \\
     & \textbf{Research technician (2015)} Länderinstitut für Bienenkunde (Berlin, Germany) \\
     & \textit{Infection of honeybees and bumblebees with Nosema ceranae, a deadly emerging parasite of bees} \\\vspace{-6pt}
     & \\
     & \textbf{Research technician (2014)} University of Liverpool (Liverpool, UK) \\
     & \textit{Triclabendazole resistance in \textit{Fasciola hepatica} (liver fluke)} \\\vspace{-6pt}
     & \\
     & \textbf{Volunteer lab/field assistant (2013)} Robert Koch Institute (Berlin, Germany and Taï forest, Ivory Coast) \\
     & \textit{“Great Ape Health Monitoring” project. Field work, DNA extraction, routine diagnostic PCR} \\\vspace{-6pt}
     & \\
     & \textbf{Intern (2012 - 2013)} National Museum of Natural History (Paris, France) \\
     & \textit{Population genetics \& morphometry of the European earwig} \\\vspace{-6pt}
     & \\
\Large{\faIcon{graduation-cap}} & \textcolor{teal}{\textbf{EDUCATION}} \\\vspace{-6pt}   
     & \\
     & \textbf{PhD in biomedical sciences (2016 - 2021)} \textit{magna cum laude}\\
     & Freie Universität Berlin (Berlin, Germany) \\
     & Thesis: \href{https://refubium.fu-berlin.de/handle/fub188/29547}{Resistance and tolerance to \textit{Eimeria} in the European house mouse hybrid zone} \\\vspace{-6pt}
     & \\
     & \textbf{MSc speciality Biology, Health, Ecology (2013)} \\
     & École Pratique des Hautes Études (Paris, France) \\\vspace{-6pt}
     & \\
     & \textbf{Veterinarian Doctorate (2008 - 2013)} \textit{First Class Honours}\\
     & École Nationale Vétérinaire d’Alfort (Maisons-Alfort, France) \\
     & Thesis: \href{https://theses.vet-alfort.fr/telecharger.php?id=1855}{Phylogeny, differentiation and speciation, example of the European earwig (\textit{Forficula auricularia})}
\end{tabular}
    
\newpage

\noindent \begin{tabular}{@{} l p{0.95\linewidth}}
\Large{ \faIcon{award}} & \textcolor{teal}{\textbf{AWARDS \& FUNDINGS}} \\\vspace{-6pt}
     & \\
     & \textbf{Associate Fellowship of the Higher Education Academy (2025)}\\
     & \\
     & \textbf{Marie Skłodowska-Curie Individual Fellowship, European Commission (2021)}\\
     & Molecular trade-offs between adaptation to salinity and immunity in three-spined sticklebacks \\\vspace{-6pt} 
     & \\
     & \textbf{Walter Benjamin Programme, Deutsche Forschungsgemeinschaft (2021)}(obtained but declined)\\
     & Molecular trade-offs between adaptation to salinity and immunity in three-spined sticklebacks \\\vspace{-6pt} 
     & \\
     & \textbf{\pounds200 student travel grant for the BSP Spring Meeting (2019)}\\
     & British Society for Parasitology \\\vspace{-6pt} 
     & \\
      & \textbf{Veterinary dissertation prize: bronze medal (2014)}\\
     & École Nationale Vétérinaire d’Alfort (Maisons-Alfort, France) \\
     & \\
\Large{\faIcon{users}} & \textcolor{teal}{\textbf{TEACHING EXPERIENCE}} \\\vspace{-6pt}     
     & \\
     & \textbf{Teaching during postdoc in UCL, UK (2025)}: BIOL0003 Introduction to Genetics (1h); BIOL0058 Core skills Y3 (1h); GENE0006 Understanding Bioinformatics Resources and Their Application Level 7 (1h) \\\vspace{-6pt} 
     & \\
     & \textbf{Teaching fellow, QMUL, UK (2023 - 2024)}: Delivered lectures on statistics, bioinformatics and fundamental biology in several modules across levels 4 to 7 ("Diversity and Ecology", "Practical molecular and cellular biology", "Research method and communication", "Coding and data science", etc.). 
     Module organiser and delivery of Missing Biological Data Team Challenge in "AI in the Biosciences" level 7 \\\vspace{-6pt} 
     & \\
     & \textbf{R for complete beginners workshop, QMUL, UK (2021 - 2023)}: 14 hours of gained teaching experience. PhD students, postdocs, and other research staff \\\vspace{-6pt} 
     & \\
     & \textbf{Attend “Train the trainer” programme, QMUL, UK  (2021)}: Postdocs training to design and deliver courses to doctoral students \\\vspace{-6pt} 
     & \\
     & \textbf{Students supervisions}\\
     & \parbox{7.0in}{2025 Supervision of a 2nd year undergraduate summer student's project, UCL, UK} \\
     & \parbox{7.0in}{2022 Supervision of MSc thesis, QMUL, UK} \\
     & \parbox{7.0in}{2016-2019 Supervision of four BSc theses, Humboldt University Berlin, Germany} \\\vspace{-6pt} 
     & \\
\Large{\faIcon{cogs}} & \textcolor{teal}{\textbf{CONTRIBUTION TO DISCIPLINE}} \\\vspace{-6pt}  
     & \\
     & \textbf{Peer reviews}\\
     & \parbox{7.0in}{Molecular Ecology (4), Marine Biology (2), Ecology \& Evolution (1), Proceedings of The Royal Society B: Biological Sciences (1); Grant review for the Czech Science Foundation} \\\vspace{-6pt}
      & \\
      & \textbf{Co-organisation of an \href{https://evoepijc.wixsite.com/epigenetics-journal}{online international journal club on epigenetics} (2021 - today)}\\\vspace{-6pt}
      & \\
      & \textbf{Guest editor for a \href{https://onlinelibrary.wiley.com/doi/toc/10.1111/(ISSN)1752-4571.potential-of-wild-populations}{special issue in Evolutionary Application}} (2024)\\\vspace{-6pt}
      & \\
      & \textbf{Conferences \&  workshops (co)organised}\\
     & \parbox{7.0in}{Symposium “The molecular carol: past, present and future of molecular tools in Conservation Biology”, ECCB Bologna (Italy)(2024)}\\\vspace{-6pt}
      & \\      
      & \parbox{7.0in}{Symposium “Epigenetics goes wild! Epigenetic diversity and the evolutionary potential of wild populations”, ESEB Congress Pragues (Czech Republic)(2022)}\\\vspace{-6pt}
      & \\
      & \parbox{7.0in}{QMUL SBBS PDRA Summer Symposium 2022, London (UK) (27-28 June 2022)} \\\vspace{-6pt}
            & \\
      & \parbox{7.0in}{Organisation of "Philosophy of science for biologists - Theory of science in relation to evolution", Berlin (Germany) \href{https://www.zibi-berlin.de/events/other/philosophy-of-science.html}{(14-15 March 2018)} and 2 x "Introduction to evolutionary biology for infection biologists", Berlin (Germany) (\href{https://www.vetmed.fu-berlin.de/en/einrichtungen/sonstige/grk2046/_downloads/0_-GRK2046-News/Evolutionary-Workshop-May-2016.pdf}{12-13 May 2016}) \& (\href{https://www.vetmed.fu-berlin.de/en/einrichtungen/sonstige/grk2046/news/ws-evolutionary-biology-2017.pdf}{15-16 May 2017)
      })} \\\vspace{-6pt}
      \end{tabular}
\newpage

\noindent \begin{tabular}{@{} l p{0.95\linewidth}}
      & \textbf{Administrative engagement}\\
      & \parbox{7.0in}{PDRA representative for the department of Biology, SBBS, QMUL, London, UK (2022)}      \\      
      & \parbox{7.0in}{Part of \href{https://www.ucl.ac.uk/biosciences/about/equality-diversity-and-inclusion/athena-swan-biosciences}{UCL Biosciences Athena SWAN Self-Assessment Team} (from 2025 on)}\\
      & \parbox{7.0in}{Co-host for \href{https://in2scienceuk.org/our-programmes/in2research/}{In2research} (2025-2026)}\\\\
     
\Large{\faIcon{bullhorn}} & \textcolor{teal}{\textbf{OUTREACH}} \\\vspace{-6pt}  
     & \\
     & \textbf{Co-organisation of \href{https://www.qmul.ac.uk/eizaguirrelab/turtle-project/wild-live-streaming/}{“Wild-Life Streaming”} (2022)}\\
     & \parbox{7.0in}{visit to UK schools to help students discover the work of international conservation NGOs and explore career pathways in biodiversity conservation} \\\vspace{-6pt}
      & \\
     & \textbf{Science Slam Berlin, (Germany) (2019 \& 2020)}\\
     & \parbox{7.0in}{Shared winner of \#75 in 2020 (\href{https://www.youtube.com/watch?v=xc3TqjdTCio}{youtube video link})} \\\vspace{-6pt}
      &\\
\Large{\faIcon{comment}} & \textcolor{teal}{\textbf{INVITED PRESENTATIONS}} \\\vspace{-6pt}  
     & \\
     & \textbf{Wilhelminenberg Seminar Talk serie, Research Institute of Wildlife Ecology, University of Veterinary Medicine Vienna (Austria)(2025)}\\
     & Talk: Understanding the role of methylation in adaptive response to environmental pressures \\\vspace{-6pt} 
     & \\
     & \textbf{Epigenetics for wildlife research workshop, Norwegian Institute for Nature Research (Norway)(2025}\\
     & Talk: An epigenetic toolbox for conservation biologists \\\vspace{-6pt} 
     & \\
     \Large{\faIcon{newspaper}} & \textcolor{teal}{\textbf{PUBLICATIONS}} \\\vspace{-18pt}  
\end{tabular}

     \begin{enumerate}

      \item Helmkampf, M., Coulmance, F., Heckwolf, M. J., Acero, A., \textbf{Balard A.}, Bista, I., Dominguez Dominguez O., Frandsen, P. B., IKMB, Santaquiteria, A., Tavera, J., Victor, B., Robertson, R., Betancur, R., McMillan, W. O., Puebla, O. (2025) Radiation with reproductive isolation in the near-absence of phylogenetic signal. \textit{Science advances}. doi: \href{https://doi.org/10.1126/sciadv.adt0973}{10.1126/sciadv.adt0973}

     \item Yen, E. C., Gilbert J. D., \textbf{Balard, A.}, Taxonera, A., Fairweather, K., Ford, H. L., Thorburn, D.-M. J., Rossiter,  S. J., Martin-Duran, J. M., Eizaguirre, C. (2025) Chromosome-level genome assembly and methylome profile enables insights for the conservation of endangered loggerhead sea turtles. \textit{Gigascience}. doi: \href{https://doi.org/10.1093/gigascience/giaf054}{10.1093/gigascience/giaf054}

     \item \textbf{Balard A.}, Baltazar-Soares, M., Eizaguirre, C., Heckwolf, M. J. (2024). An epigenetic toolbox for conservation biologists. \textit{Evolutionary Applications}. doi: \href{https://doi.org/10.1111/eva.13699}{10.1111/eva.13699}

     \item Baltazar-Soares, M., \textbf{Balard A.}, Heckwolf, M. J. (2024). Epigenetic diversity and the evolutionary potential of wild populations. \textit{Evolutionary Applications}. doi: \href{ https://doi.org/10.1111/eva.70011}{10.1111/eva.70011}

     \item Yen, E., Gilbert, J., \textbf{Balard A.}, Afonso, I., Fairweather, K., Newlands, D., Lopes, A., Correia, S., Taxonera, A., Rossiter, S., Martín‑Durán, J., Eizaguirre, C. (2024) DNA methylation carries signatures of sublethal effects under thermal stress in loggerhead sea turtles. \textit{Evolutionary Applications}. doi: \href{https://doi.org/10.1101/2023.11.22.568239}{10.1101/2023.11.22.568239}   
      
     \item Jarquín-Díaz, V. H., Ferreira, S. C. M., \textbf{Balard A.}, Ďureje, Ľ., Macholán, M., Piálek, J., Bengtsson-Palme J., Kramer-Schadt, S., Forslund, S. K., Heitlinger, E. (2024). Aberrant microbiomes are associated with increased antibiotic resistance gene load in hybrid mice. \textit{ISME Communication}. doi: \href{https://apps.crossref.org/pendingpub/pendingpub.html?doi=10.1093%2Fismeco%2Fycae053}{10.1093/ismeco/ycae053}
     
     \item \textbf{Balard A.}, \& Heitlinger, E. (2022) Shifting focus from resistance to disease tolerance: a review on hybrid house mice. \textit{Ecology and Evolution}. doi: \href{https://doi.org/10.1002/ece3.8889}{10.1002/ece3.8889}
     
     \item Jarquín-Díaz, V. H., \textbf{Balard A.}, Ferreira, S. C. M., Mittné V., Murata J. M., Heitlinger E. (2022) DNA-based quantification and counting of transmission stages provides different but complementary parasite load estimates: an example from rodent coccidia (\textit{Eimeria}). \textit{Parasites \& Vectors}. doi: \href{https://doi.org/10.1186/s13071-021-05119-0}{10.1186/s13071-021-05119-0}
     
    \item \textbf{Balard A.}, Jarquín-Díaz, V. H., Jost, J., Mittné, V., Böhning, F., Ďureje, Ľ., Piálek, J., Heitlinger, E. (2020) Coupling between tolerance and resistance for two related \textit{Eimeria} parasite species. \textit{Ecology \& Evolution}. doi: \href{https://doi.org/10.1002/ece3.6986}{10.1002/ece3.6986}

     \item  Jarquín-Díaz, V. H., \textbf{Balard A.}, Mácová, A., Jost, J., Roth von Szepesbéla, T., Berktold, K., Tank, S., Kvičerová, J., Heitlinger, E. (2020) Generalist \textit{Eimeria} species in rodents: multilocus analyses indicate inadequate resolution of established markers. \textit{Ecology \& Evolution}. doi: \href{https://doi.org/10.1002/ece3.5992}{10.1002/ece3.5992}

    \item  \textbf{Balard A.}, Jarquín-Díaz, V. H., Jost, J., Martincová, I., Ďureje, Ľ., Piálek, J., Macholán, M., Goüy de Bellocq, J., Baird, S. J. E., Heitlinger, E. (2019) Intensity of infection with intracellular \textit{Eimeria} spp. and pinworms is reduced in hybrid mice compared to parental subspecies. \textit{Journal of Evolutionary Biology}. doi: \href{https://doi.org/10.1111/jeb.13578}{10.1111/jeb.13578}

    \item   Jarquín-Díaz, V. H., \textbf{Balard A.}, Jost, J., Kraft, J., Dikmen, M. N., Kvičerová, J., Heitlinger, E. (2019) Detection and quantification of house mouse \textit{Eimeria} at the species level-challenges and solutions for the assessment of Coccidia in wildlife. \textit{Int J Parasitol Parasites Wildl}. doi: \href{https://doi.org/10.1016/j.ijppaw.2019.07.004}{10.1016/j.ijppaw.2019.07.004}

    \item   Bredtmann, C. M., Krücken, J., Murugaiyan, J, \textbf{Balard A.}, Hofer, H., Kuzmina, T. A., von Samson-Himmelstjerna, G. (2019) Concurrent proteomic fingerprinting and molecular analysis of cyathostomins. \textit{Proteomics}. doi: \href{https://doi.org/10.1002/pmic.201800290}{10.1002/pmic.201800290}
     
     \end{enumerate}
   
\noindent \begin{tabular}{@{} l p{0.95\linewidth}}
\Large{\faIcon{pen-nib}} & \textcolor{teal}{\textbf{ARTICLE IN PREPARATION}} \\\vspace{-18pt} 
\end{tabular}

\begin{itemize}

\item \textbf{Balard A.} \& Strassert J. F. H., Wolinska J., Martinez Ruiz E. Herbicide metolachlor alters gene expression and influences the interaction between a bloom-forming cyanobacterium and its chytrid parasite. biorxiv. doi: \href{https://doi.org/10.1101/2025.10.07.680865}{10.1101/2025.10.07.680865} Submitted to ISME

\item \textbf{Balard A.}, Sagonas, K., Kaufmann, J., Heckwolf, M. J., Eizaguirre, C. DNA methylation provides a molecular basis for disease tolerance and intergenerational paternal effects. in prep 

\item Lee J.* \& \textbf{Balard A.*}, Lacerda E., Gilbert J., Kingdon C., Abken E., Dockrell H., Cliff J., Nacul L. Immunilogical markers of chronic fatigue. in prep
\end{itemize}

\end{document}
